 \chapter{Разработка моделей и алгоритмов }

Булева функция --- функция вида $B^{n} \rightarrow B$, где $B = {0,1}$. 
Систему булевых функций называют функционально полной, если из ее набора 
функций можно представить любую булеву функцию. Теорема (критерий) Поста 
позволяет проверить является ли система функций полной. Идея теоремы состоит 
в том, чтобы рассматривать множество всех булевых функций PA как алгебру 
относительно операции 
суперпозиции. Данная алгебра называется алгеброй Поста. Эта алгебра содержит в 
качестве своих подалгебр множества функций, замкнутых относительно 
суперпозиции. Их называют ещё замкнутыми классами. Их называют ещё замкнутыми 
классами. Пусть R --- некоторое подмножество PA. Замыканием [R] множества  R 
называется минимальная подалгебра PA, содержащая R. Иными словами, замыкание 
состоит из всех функций, которые являются суперпозициями R. Очевидно, что R 
будет функционально полно тогда и только тогда, когда [R]=PA. Таким образом, 
вопрос, будет ли данный класс функционально полон, сводится к проверке того, 
совпадает ли его замыкание с PA.


Оператор [ \_ ] является оператором замыкания. Он обладает следующими свойствами:

\begin{itemize}
	\item $R \subseteq [R]$
	\item $R_{1} \subseteq R_{2} \Rightarrow [R_{1}] \subseteq [R_{2}]$ 
	\item $[[R]] = [R]$
\end{itemize}

Пост сформулировал необходимое и достаточное условие полноты системы булевых 
функций. Для этого он ввел в рассмотрение следующие замкнутые классы булевых 
функций: функции, сохраняющие константу T_{0} и T_{1}, самодвойственныые функции S,
монотонные функции M, линейные функции L. Введем определения данных классов функций.

\textit{Класс функций сохраняющих ноль T_{0}}: говорят, что функция сохраняет ноль, если $f(0, 0, \ldots , 0) = 0$.

\textit{Класс функций сохраняющих единицу T_{1}}: говорят, что функция сохраняет единицу, если $f(1, 1, \ldots , 1) = 1$.

\textit{Класс самодвойственных функций S}: говорят, что функция самодвойственна , если $f(\overline{x_{1}}, \ldots , \overline{x_{n}})=\overline{f(x_{1},\ldots,x_{n})}$. Иными словами, функция называется самодвойственной, если на противоположных наборах она принимает противоположные значения.

\textit{Класс монотонных функций M}: говорят, что функция монотонна, если $\forall i (a_{i} \leqslant b_{i}) \Rightarrow f(a_{1}, \ldots ,a_{n})\leqslant f(b_{1}, \ldots ,b_{n})$.

\textit{Класс линейных функций L}: говорят, что функция линейна , если существуют такие $a_{0}, a_{1}, a_{2}, \ldots, a_{n}$, где $a_{i} \in \{0, 1\}, \forall i=\overline{1,n}$, что для любых $x_{1}, x_{2}, \ldots, x_{n}$ имеет место равенство:
$f(x_{1}, x_{2}, \ldots, x_{n}) = a_{0} \oplus a_{1} \cdot x_{1} \oplus a_{2} \cdot x_{2} \oplus \ldots \oplus a_{n} \cdot x_{n}$.